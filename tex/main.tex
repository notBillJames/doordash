\documentclass{article}
\usepackage[utf8]{inputenc}
\usepackage{amsmath}
\usepackage{titling}

\title{DoorDash}
\author{Charles Lyman}
\date{}

\begin{document}

\maketitle

\section*{The Problem}
    The pay $P$ that an employee receives for their work can be modeled as a function of their time worked in hours, $t$, and the rate they are paid, $r$ looking something like this
    
        $$P=rt$$
        
\noindent When doordashing however, the rate $r$ varies between orders. For a doordasher the money $P$ a dasher makes is a sum of all the orders done in the time dashing. The pay of an order in dollars, $P_o$, is similarly modeled as a function of the order rate in dollars per hour, $r_o$, and order time in hours, $t_o$ as follows

        $$P_o=r_o*t_o$$
	
\noindent If I am going to assume that $r_o$ varies between orders, then it would also be reasonable to assume that the graph of the frequency of different values of $r_o$ would resemble something of a bell curve. In order to maximize the pay they receive, a dasher would want to take orders with the highest pay rate and minimum wait time. The question that I hope to asnwer is what minimum value $r_o$ should a dasher to set to make the most money possible while working?

\section*{Understanding an Order}

There are really on 3 important inputs to an order and they are:
\begin{enumerate}
    \item Pay
    \item Restaurant
    \item Destination
\end{enumerate}

\noindent Those 3 inputs result in a few different qualities of an order, namely
\begin{enumerate}
    \item Distance
    \item Pickup Time
    \item Delivery Time
    \item Avg. Speed
    \item Dollars / Mile
\end{enumerate}
\noindent I think it's important to understand the inputs and qualities of each data point in order to enhance the analysis

\section*{Modeling $r_o$}
If the pay rate, r, is measured in dollars per hour then r will increase as dollars increases and hours decreases. 
\\
\\
\noindent Wage rate could be calculated as:

	$$r_o= \frac{dollars}{mile}* \frac{miles}{hour}$$
	
\noindent An additional factor to consider is the frequency of orders with optimal values of $r_o$. This is because time spent working, $t$, can be written as \\ 
$t=(hours\;dashing - hours\;waiting)$ and $hours\;waiting$ certainly will increase as we wait for higher values of $r$. This makes sense as $r$ and $t$ should have an inverse relationship in the equation $P=rt$. \\
\\
\noindent So, Pay could be modeled as:

$$P=\left(\frac{dollars}{mile}*\frac{miles}{hour}\right)*(hours\;dashing - hours\;waiting)$$

\section*{Accounting for Varying Frequencies of $r_o$}

Orders will come with varying rates of  $r_o$. It is natural to assume that higher values of  $r_o$  have lower frequencies and lower values of  $r_o$  have higher frequencies. To be able to make the best selection of orders, I need to find the range of values for  $r_o$  that return the highest net pay, P. In short, I do not want to spend time doing low-paying orders when I could be doing high-paying orders but I also do not want to be waiting for high-paying orders when I could be making money with low-paying orders.
\newline
\newline
\noindent Suppose there are two orders $O_1$ and $O_2$ with different values of $r_o$, and similar times to complete each order.  The pay of both models could be modeled like this, 
\newline \newline
\begin{align*}
    P_1&=(O_1)*(hours\;dashing - hours\;waiting) \\
    P_2&=(O_2)*(hours\;dashing - hours\;waiting)
\end{align*}

Higher order frequency results in less time waiting, and more time working. Assuming that order frequency takes the shape of a bell curve as a function of  $r_o$  then we could say that the time spent working, t, is equal to this equation:

$$t = \left(\frac{orders}{hour}\right) * t_o$$

\noindent where $t_o$  is the time required to complete the order in hours. I should note that because everything but $r_o$ is constant in my example right now in my example, one order can be interpreted as a fixed unit of time, x hours. So the equation above can be read as x hours working / hours waiting * hours dashing
\newline
\newline
The next step is modeling frequency of orders according to values of $r_o$, and the time requirement for an order given the pickup and drop off location




\end{document}
